% Stewart Dulaney
% https://www.stewartdulaney.com
% This document was adapted from the templates posted at the following sources:
% https://www.cs.cmu.edu/~ckingsf/class/02-714/hw-template.tex
% http://www.math-cs.gordon.edu/courses/mat231/handouts/truth-table-latex.tex
%
\documentclass[11pt]{article}
\usepackage{amsmath,amssymb,amsthm}
\usepackage{graphicx}
\usepackage[margin=1in]{geometry}
\usepackage{fancyhdr}
\setlength{\parindent}{0pt}
\setlength{\parskip}{5pt plus 1pt}
\setlength{\headheight}{13.6pt}
\newcommand\question[2]{\vspace{.25in}\hrule\textbf{#1: #2}\vspace{.5em}\hrule\vspace{.10in}}
\renewcommand\part[1]{\vspace{.10in}\textbf{(#1)}}
\newcommand\answer{\vspace{.10in}\textbf{Answer: }}
\pagestyle{fancyplain}
\lhead{\textbf{\NAME\ (SID: \SID)}}
\chead{\textbf{HW\HWNUM}}
\rhead{MATH 10, \today}
\begin{document}\raggedright
%Section A==============Change the values below to match your information==================
\newcommand\NAME{Stewart Dulaney}  % your name
\newcommand\SID{1545566}     % your smc student id
\newcommand\HWNUM{2}              % the homework number
%Section B==============Put your answers to the questions below here=======================

\question{1}{Prove $x$ is odd if and only if $x^2 + 6x + 9$ is even.} 

Strategy 1:  Since this problem is of the form $P \Leftrightarrow Q$. You must show $P \Rightarrow Q$ and $Q \Rightarrow P$.

Strategy 2:  Using contrapositives (pg 8) You can instead prove $\lnot P \Leftrightarrow \lnot Q$ by showing $\lnot P \Rightarrow \lnot Q$ and $\lnot Q \Rightarrow \lnot P$.

Strategy 3:  You can sometimes start with a known biconditional and connect ideas: $P \Leftrightarrow A_1$ but $A_1 \Leftrightarrow A_2$ etc...... $A_k \Leftrightarrow Q$ thus $P \Leftrightarrow Q$.

\answer (using Strategy 1)

Let $P =$ "$x$ is odd" and $Q =$ "$x^2 + 6x + 9$ is even".

\part{Step 1} Show $P \Rightarrow Q$

We assume that $P$ is true. By the definition of an odd integer, it follows that $x = 2k + 1$, where $k$ is some integer.
This implies that:\\
\begin{align*}
  x^2 + 6x + 9 &= (2k + 1)^2 + 6(2k + 1) + 9 \\
  &= (2k + 1)(2k + 1) + 12k + 6 + 9 \\
  &= (4k^2 + 2k + 2k + 1) + 12k + 15 \\
  &= 4k^2 + 16k + 16 \\
  &= 2(2k^2 + 8k + 8) \\
\end{align*}

Because $x^2 + 6x + 9$ is $2t$, where $t$ is some integer $2k^2 + 8k + 8$, $x^2 + 6x + 9$ is even. This proves $P \Rightarrow Q$.

\part{Step 2} Show $Q \Rightarrow P$

We use a proof by contraposition and show $\lnot P \Rightarrow \lnot Q$. We assume $\lnot P$, namely, that $x$ is even. By the definition of an even integer, it follows that $x = 2m$, where $m$ is some integer. 
This implies that:\\
\begin{align*}
  x^2 + 6x + 9 &= (2m)^2 + 6(2m) + 9 \\
  &= 4m^2 + 12m + 9 \\
  &= 4m^2 + 12m + 8 + 1 \\
  &= 2(2m^2 + 6m + 4) + 1 \\
\end{align*}

Because $x^2 + 6x + 9$ is $2n + 1$, where $n$ is some integer $2m^2 + 6m + 4$, $x^2 + 6x + 9$ is odd. This proves $\lnot P \Rightarrow \lnot Q$, which tells us that $Q \Rightarrow P$ is also true. \\[\baselineskip]

Because we have shown that both $P \Rightarrow Q$ and $Q \Rightarrow P$ are true, we have shown that $P \Leftrightarrow Q$ is true.

\end{document}
