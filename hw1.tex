% Stewart Dulaney
% https://www.stewartdulaney.com
% This document was adapted from the templates posted at the following sources:
% https://www.cs.cmu.edu/~ckingsf/class/02-714/hw-template.tex
% http://www.math-cs.gordon.edu/courses/mat231/handouts/truth-table-latex.tex
%
\documentclass[11pt]{article}
\usepackage{amsmath,amssymb,amsthm}
\usepackage{graphicx}
\usepackage[margin=1in]{geometry}
\usepackage{fancyhdr}
\setlength{\parindent}{0pt}
\setlength{\parskip}{5pt plus 1pt}
\setlength{\headheight}{13.6pt}
\newcommand\question[2]{\vspace{.25in}\hrule\textbf{#1: #2}\vspace{.5em}\hrule\vspace{.10in}}
\renewcommand\part[1]{\vspace{.10in}\textbf{(#1)}}
\newcommand\answer{\vspace{.10in}\textbf{Answer: }}
\pagestyle{fancyplain}
\lhead{\textbf{\NAME\ (SID: \SID)}}
\chead{\textbf{HW\HWNUM}}
\rhead{MATH 10, \today}
\begin{document}\raggedright
%Section A==============Change the values below to match your information==================
\newcommand\NAME{Stewart Dulaney}  % your name
\newcommand\SID{1545566}     % your smc student id
\newcommand\HWNUM{1}              % the homework number
%Section B==============Put your answers to the questions below here=======================

\question{1}{Negate each of the following statements:} 

\part{a} It is Monday and it is snowing.

\answer It is not Monday or it is not snowing.

\part{b} This book is red or it was written in 1887.

\answer This book is not red and it was not written in 1887.

\part{c} If I tell you a joke, you will smile.

\answer I tell you a joke and you do not smile.

\question{2}{Let X="Fred has red hair.",  Y="Fred has freckles.", and Z="Fred likes to eat figs.".}

\part{a} "Fred has red hair, and does not have freckles."

\answer $X \land \lnot Y$

\part{b} "Fred likes to eat figs, and he has red hair or he has a freckles."

\answer $Z \land (X \lor Y)$

\part{c} "It is not the case that Fred has freckles or he has red hair."

\answer $\lnot (Y \lor X)$

\part{d} "It is not the case that Fred has freckles, or he has red hair."

\answer $(\lnot Y) \lor X \equiv \lnot Y \lor X$

\clearpage

\question{3}{Make a truth table for: $(P \rightarrow R) \lor \lnot(Q)$}

\answer

\begin{displaymath}  % start unumbered math environment
  %
  % Start a table in math mode.  The |c|c|c|c|c|c|c|c| string is a
  % format string that says there will be 8 colunms in the table.  The
  % c's indicate that the data in each column will be centered (use l
  % for left justified and r for right justified).  The vertical bar
  % means that lines will be drawn between columns.  The trailing
  % \hline causes a horizontal line to be drawn across the top of the
  % table.
  %
  \begin{array}{|c|c|c|c|c|c|c|c|}\hline
    %
    % Each row of the table consists of data separated by "&" symbols.
    % Each row must end with "\\" to cause a newline.  A trailing
    % \hline will cause a line to be drawn under the row.  A double
    % \hline is often used to separate the table header from the rest
    % of the table. 
    %
    P & Q & R & P \rightarrow R & \lnot Q & (P \rightarrow R) \lor \lnot Q\\\hline\hline
    T & T & T & T & F & \mathbf{T}\\\hline
    T & T & F & F & F & \mathbf{F}\\\hline
    T & F & T & T & T & \mathbf{T}\\\hline
    T & F & F & F & T & \mathbf{T}\\\hline
    F & T & T & T & F & \mathbf{T}\\\hline
    F & T & F & T & F & \mathbf{T}\\\hline
    F & F & T & T & T & \mathbf{T}\\\hline
    F & F & F & T & T & \mathbf{T}\\\hline
  \end{array}
\end{displaymath}

\end{document}
