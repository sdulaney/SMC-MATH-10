% Stewart Dulaney
% https://www.stewartdulaney.com
% This document was adapted from the templates posted at the following sources:
% https://www.cs.cmu.edu/~ckingsf/class/02-714/hw-template.tex
% http://www.math-cs.gordon.edu/courses/mat231/handouts/truth-table-latex.tex
%
\documentclass[11pt]{article}
\usepackage{amsmath,amssymb,amsthm,mathabx}
\usepackage{graphicx}
\usepackage[margin=1in]{geometry}
\usepackage{fancyhdr}
\setlength{\parindent}{0pt}
\setlength{\parskip}{5pt plus 1pt}
\setlength{\headheight}{13.6pt}
\newcommand\question[2]{\vspace{.25in}\hrule\textbf{#1: #2}\vspace{.5em}\hrule\vspace{.10in}}
\renewcommand\part[1]{\vspace{.10in}\textbf{(#1)}}
\newcommand\answer{\vspace{.10in}\textbf{Answer: }}
\pagestyle{fancyplain}
\lhead{\textbf{\NAME\ (SID: \SID)}}
\chead{\textbf{HW\HWNUM}}
\rhead{MATH 10, \today}
\begin{document}\raggedright
%Section A==============Change the values below to match your information==================
\newcommand\NAME{Stewart Dulaney}  % your name
\newcommand\SID{1545566}     % your smc student id
\newcommand\HWNUM{10}              % the homework number
%Section B==============Put your answers to the questions below here=======================

\question{1}{Purpose to show why there are two parts of induction steps required.\\[\baselineskip]Given the statement $P(n) =$ "$10^n$ is divisible by $7$"}

\part{a} Prove that $P(n) \rightarrow P(n + 1)$ is a tautology.\\[\baselineskip]
Comments. You are doing step 2 first!\\[\baselineskip]
Case 1: $P(n)$ is false for all nonnegative integers $n$ then the conditional will always be true.\\[\baselineskip]
Case 2: $P(n)$ is true for some nonnegative integer $n$.\\[\baselineskip]
YOU FILL IN THE REST and explain why $P(n + 1)$ must also be true.

\part{b} Prove that $P(n)$ is not true for any nonnegative integer.

\part{c} Do the results in part (a) and part (b) contradict the principle of mathematical induction? Explain.

\end{document}
